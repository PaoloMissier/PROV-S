% % % % % % % % % % %
%-----------------------------------------------------------------------------
%
%               Template for sigplanconf LaTeX Class
%
% Name:         sigplanconf-template.tex
%
% Purpose:      A template for sigplanconf.cls, which is a LaTeX 2e class
%               file for SIGPLAN conference proceedings.
%
% Guide:        Refer to "Author's Guide to the ACM SIGPLAN Class,"
%               sigplanconf-guide.pdf
%
% Author:       Paul C. Anagnostopoulos
%               Windfall Software
%               978 371-2316
%               paul@windfall.com
%
% Created:      15 February 2005
%
%-----------------------------------------------------------------------------
\documentclass[preprint]{sigplanconf}

\usepackage{amsmath}

%\usepackage{biblatex}
%\usepackage[style=ieee]{biblatex}
%\AtEveryBibitem{%
%  \clearfield{issn} % Remove issn
%  \clearfield{doi} % Remove doi
%
%  \ifentrytype{online}{}{% Remove url except for @online
%    \clearfield{url}
%  }
%}

%\usepackage{pifont}
%\usepackage[T1]{fontenc} % this is just to get the natural encoding for the < > characters in text

\usepackage{color,soul}
\definecolor{highlight}{rgb}{1,1,0.6}
\definecolor{link}{rgb}{0.5,0.0,0.0}
\definecolor{cite}{rgb}{0.0,0.0,0.6}
\definecolor{url} {rgb}{0.3,0.0,0.3}
%\definecolor{grey}{rgb}{0.3,0.3,0.3}

\usepackage{algorithm}
\usepackage{algpseudocode}
\usepackage{listings}

\usepackage[hidelinks]{hyperref}
\hypersetup{
    colorlinks,
    linkcolor={cite},
    citecolor={cite},
    urlcolor ={cite}
}
\usepackage{enumitem}
\usepackage[british]{datetime2}

\usepackage{tabularx}
%\usepackage{array}
\usepackage{multirow}
\usepackage{booktabs}
%\newcolumntype{L}{>{\raggedright\arraybackslash}p{3cm}}
%\usepackage{dcolumn}
%\usepackage{multicol}

% % annotations environments % % 
\newcommand{\note}[1]{\textit{\textcolor{red}{\{#1\}}}}
\sethlcolor{highlight}


\def\ojoin{\setbox0=\hbox{$\bowtie$}%
  \rule[-.02ex]{.25em}{.4pt}\llap{\rule[\ht0]{.25em}{.4pt}}}
\def\leftouterjoin{\mathbin{\ojoin\mkern-5.8mu\bowtie}}
\def\rightouterjoin{\mathbin{\bowtie\mkern-5.8mu\ojoin}}
\def\fullouterjoin{\mathbin{\ojoin\mkern-5.8mu\bowtie\mkern-5.8mu\ojoin}}


%% PM Define authornote command for comments
\newcommand{\authornote}[1] {
    \begin{center}
        \framebox{
            {\begin{minipage}[t]{0.9\linewidth}
                \raggedright  \textbf{[PM]}~ \scriptsize #1 \normalsize
            \end{minipage}}
    }
    \end{center}
}

\usepackage{amssymb}
\let\oldemptyset\emptyset
\let\emptyset\varnothing

\usepackage{bm}
\usepackage{mathtools}
\def\filtcap{\mathrel{%
    \mathchoice{\FILTCAP}{\FILTCAP}{\scriptsize\FILTCAP}{\tiny\FILTCAP}%
}}
\def\FILTCAP{{%
    \setbox0\hbox{$\cap$}%
    \rlap{\hbox to \wd0{\hss$\circ$\hss}}\box0
}}


\newcommand{\structurenote}[2] {
    \begin{center}
        \framebox{
            \colorbox{purple!40}{\begin{minipage}[t]{0.9\linewidth}
                \raggedright  \textbf{[#1]}~ \scriptsize #2 \normalsize
            \end{minipage}}
    }
    \end{center}
}


%\usepackage{datetime}
%\usepackage[mark]{gitinfo2}

\newcommand{\prov}{\mathit{prov}}
\newcommand{\provs}{\texttt{PROV-S}}
\newcommand{\entity}{\texttt{entity}}
\newcommand{\cL}{{\cal L}}
\newcommand{\os}{\mathit{os}}
\newcommand{\is}{\mathit{i}s}



% The following \documentclass options may be useful:

% preprint       Remove this option only once the paper is in final form.
%  9pt           Set paper in  9-point type (instead of default 10-point)
% 11pt           Set paper in 11-point type (instead of default 10-point).
% numbers        Produce numeric citations with natbib (instead of default author/year).
% authorversion  Prepare an author version, with appropriate copyright-space text.


\begin{document}
%\lstset{language=Pyton}

\special{papersize=8.5in,11in}
\setlength{\pdfpageheight}{\paperheight}
\setlength{\pdfpagewidth}{\paperwidth}

\conferenceinfo{TAPP'17}{June 22-23, 2017 Seattle, WA, USA}
\copyrightyear{2017}
\copyrightdata{978-1-nnnn-nnnn-n/yy/mm}\reprintprice{\$15.00}
\copyrightdoi{nnnnnnn.nnnnnnn}

% For compatibility with auto-generated ACM eRights management
% instructions, the following alternate commands are also supported.
%\CopyrightYear{2016}
%\conferenceinfo{CONF'yy,}{Month d--d, 20yy, City, ST, Country}
%\isbn{978-1-nnnn-nnnn-n/yy/mm}\acmPrice{\$15.00}
%\doi{http://dx.doi.org/10.1145/nnnnnnn.nnnnnnn}

% Uncomment the publication rights used.
%\setcopyright{acmcopyright}
\setcopyright{acmlicensed}  % default
%\setcopyright{rightsretained}

%\titlebanner{banner above paper title}        % These are ignored unless
%\preprintfooter{short description of paper}   % 'preprint' option specified.

\title{Provenance for data streaming based on Templates}

\authorinfo{Paolo Missier}
           {School of Computing Science \\ Newcastle University, UK}
           {paolo.Missier@ncl.ac.uk}
\authorinfo{Vasa Curcin}
           {King's College London, UK}
           {vasa.curcin@kcl.ac.uk}

\maketitle

\begin{abstract}
Data Scientists insist that valuable knowledge can be extracted from primary data sources through complex data analytics applications.
Metadata experts tend to agree, but they also point out that capturing the detailed provenance of such knowledge is key to establishing its trust, and thus to asserting its value --- be it scientific or commercial.
However, two elements combine to making the systematic recording of detailed provenance challenging in an analytics setting.
Firsty, complex analytics pipelines tend to be opaque to the detailed observation of their internal workings, which is a condition for recording the provenance of its outcomes. Secondly, much of the interesting primary data for analytics increasingly comes in the form of streams, for instance from sensors and Internet of Things devices, 
and yet provenance recorders are not generally designed to operate on streams.
In this short paper we propose that both issues can be addressed in a principled way, using a combination of provenance templates and simple meta-processes that  instantiate the templates when are executed alongside the main analytics.
Using these mechanisms, we show that the provenance of the outputs from analytics that operate on streams is itself naturally represented as a stream of provenance assertions.
\end{abstract}

% 2012 ACM Computing Classification System (CSS) concepts
% Generate at 'http://dl.acm.org/ccs/ccs.cfm'.
\begin{CCSXML}
%<ccs2012>
%<concept>
%<concept_id>10011007.10011006.10011008</concept_id>
%<concept_desc>Software and its engineering~General programming languages</concept_desc>
%<concept_significance>500</concept_significance>
%</concept>
%<concept>
%<concept_id>10003752.10010124.10010138.10010143</concept_id>
%<concept_desc>Theory of computation~Program analysis</concept_desc>
%<concept_significance>300</concept_significance>
%</concept>
%</ccs2012>
\end{CCSXML}

%\ccsdesc[500]{Software and its engineering~General programming languages}
%\ccsdesc[300]{Theory of computation~Program analysis}
% end generated code

% Legacy 1998 ACM Computing Classification System categories are also
% supported, but not recommended.
%\category{CR-number}{subcategory}{third-level}[fourth-level]
%\category{D.3.0}{Programming Languages}{General}
%\category{F.3.2}{Logics and Meanings of Programs}{Semantics of Programming Languages}[Program analysis]

\keywords
keyword1, keyword2



\section{Introduction}
	
\authornote{
\textbf{Goal: }we want to characterise provenance where entities are multiple data streams and activities are aggregators over the streams.

\textbf{Motivation:} Data in movement, where \textit{Velocity} is one of the well-known 4 V's of ``big data``, is gaining prominence as a prime source of data for analytics applications.

Example of relevant application: tracking flows of (personal) data from IoT devices. An increasingly pervasive form of moving data. Tracking is important because it enables forms of reward / credit by traversing back one or more derivation steps \note{see Paolo's IJDC paper on transitive credit}

... or choose any IoT-related example...\\

We aim to provide a mechanism for continually generating provenance assertions that account for the processing of an input data stream (more generally, a collection of input streams) through a black box transformation.


The main contribution of the (short) paper is  a framework for expressing provenance of streaming data as it goes through analytics machines (ie aggregations operators). 
}

\authornote{
\textbf{a few points to cover:}
\begin{itemize}
\item the framework deals with both stateless and stateful transformations \note{this is a fairly standard distinction, which Spark streaming makes very clearly}
\item aggregation is a form of data transformation that destroys data identity. after the transformation the data is completely new, but its lineage should contain enough info to allow for the proportion of each input stream to be calculated. 
\end{itemize}
}


\authornote{
\textbf{Sketch for a formalisation:}
transformations can be either \textbf{stateless or stateful}.  Stateless only depend deterministically on the input stream, while stateful transformations (1) depend on internal state of the activity, and (2) may change the state. An example is a state that holds the “average daily temp”. The input is a stream of temp values through one day, the transformation filters out those that are above average by more than 2 std deviations. The surviving data points are used to update the internal daily average at the end of the day.
}

\subsection{Related work}  \label{sec:related}

\note{Vasa??}

\section{Provenance generation framework}

To introduce our framework, we start from the case of a batch, ``black box'' process P, for example a service or software component with an interface that defines $n$ inputs and $m$ outputs, but unknown internal behaviour.
We use function call notation to denote an invocation of $P$:
\begin{equation}
\langle y_1 \dots y_m \rangle = P(x_1, \dots x_n)
\label{eq:P}
\end{equation}
As our initial example we use a process $P$ that takes a single, list-valued input $x = [x_1 \dots x_n]$ and produces one list-valued output $y = [y_1 \dots y_n]$ of the same size:
\begin{equation}
\langle [y_1 \dots y_n] \rangle = P([x_1 \dots x_n])
 \label{eq:P-inv}
\end{equation}
Unless something is known about the internal workings of $P$, the provenance of the $y_i$ is unknown. 
In fact, by default we cannot even claim derivation relationships of the form: \verb|y_i wasDerivedFrom x_j|.

We propose a simple provenance generation framework that enables the provenance of the $y_j$ to be asserted, despite $P$ being a black box.
Furthermore, later in this short paper we show that the framework also captures the case where $P$ computes continuously when its inputs are data streams. 

\subsection{General framework}

The framework consists of the following elements.\\

\noindent \textbf{1. A provenance template.} This is a PROV document $T$ where provenance variables may appear in lieu of constant elements, namely the identifiers for some of the Entities, Activities, and Agents, as well as some of their properties.
The following is an example of provenance template:
\begin{quote}
$\mathit{T1}$: \\
\verb|entity(f, [prov:type = `prov:plan'])| \\
\verb|entity(:x)|\\
\verb|entity(:y)| \\
\verb|activity(:a), wasAssociatedWith(:a,_,f)| \\
\verb|used(:a,:x), wasGeneratedBy(:y,:a)| \\
\verb|wasDerivedFrom(:y,:x)|
\end{quote}
We use ':x' to indicate that 'x' should be interpreted as a variable.\\

\noindent \textbf{2. A Binding.} This is a set $B =  \{ b_1 \dots b_k \}$ of assignments of constant values to variables.
An assignment $b \in B $ has the form 
\[b = \langle \mathtt{:v_1} \leftarrow \mathtt{c_1}, \mathtt{:v_2} \leftarrow \mathtt{c_2} \dots \rangle\]
and defines a substitution of the variables $\mathtt{:v_i}$ to valid PROV constants.\\

\noindent \textbf{3. A binding-generation process $P^B$,} which we refer to as a \textit{sidecar} process, which generates a binding $B$ for each invocation of $P$ using the invocation's inputs and outputs:
\begin{equation}
B = P^B(\langle \mathtt{P~inputs} \rangle, \langle \mathtt{P~outputs} \rangle)
\label{eq:B}
\end{equation}

For example, given invocation (\ref{eq:P-inv})  and the variables defined in template $T$ above, $P^B$ might generate the following binding (which we express intensionally using set notation for conciseness):
\[ \mathit{B1} = \{ \langle \texttt{:a} \leftarrow \texttt{gen}(\texttt{a},i), \texttt{:x} \leftarrow x_i, \texttt{:y} \leftarrow y_i  \rangle | i:1 \dots n\}\]
Here we have used the convenience token generator \texttt{gen(a,$i$)}, which simply generates a new constant from a seed, \texttt{a}, and a variable $i$, for instance \texttt{a\_i}.

Note that in this binding, values $x_i, y_i$ in the same position $i$ are simultaneously assigned to variables \texttt{:x}, \texttt{:y}, along with a new constant formed from seed token \texttt{a}.
Note also that what is actually assigned to a variable is the Id of a PROV Entity that represent the data, i.e., $x_i$ will be represented by an Entity and given some unique identifier. For simplicity of notation we leave this id-generation step implicit.

\noindent \textbf{4. A function $\apply(B,T)$} that generates a PROV document by applying $B$ to $T$.
Specifically:
for provenance statement $s$ in $T$,  a \textit{ground} version of $s$ for  assignment $b$, denoted $s/b$, is obtained by lexically replacing each occurrence of variable 
$\mathtt{:v}$ in $s$ by value $\mathtt{c}$ whenever $\mathtt{:v} \leftarrow \mathtt{c}$ is defined in $b$.

$\apply(B,T)$ generates the set of all statements $s/b$ for each $s \in T$ and for each $b \in B$.
Note that, by iterating over each $b \in B$, this process may generate multiple ground versions of some of the statements in $T$.\footnote{It may also be the case that the \textit{same} ground statement can be generated more than once. In this case we simply assume that only one instance is kept.}
For instance, $\apply(\mathit{B1},\mathit{T1})$ takes template $T1$ and binding $B1$ in our earlier example, and produces the following PROV document:
\begin{quote}
$P1$: \\
\verb|entity(f, [prov:type = `prov:plan'])| \\
\verb|activity(a_1), wasAssociatedWith(a_1, _, f)| \\
\verb|...| \\
\verb|activity(a_n), wasAssociatedWith(a_n, _, f)|\\
\verb|entity(x_1), ..., entity(x_n)| \\
\verb|entity(y_1), ... ,entity(y_n)| \\
\verb|used(a_1, x_1), wasGeneratedBy(y_1, a_1)| \\
\verb|...| \\
\verb|used(a_n, x_n), wasGeneratedBy(y_n, a_n)| \\
\verb|wasDerivedFrom(y_1,x_1)| \\
\verb|...| \\
\verb|wasDerivedFrom(y_n,x_n)| 
\label{ex:map}
\end{quote}

One can recognise that $P1$ is in fact a representation of the repeated application of a function $f$ to each element of input list $[x_1 \dots x_n]$, producing output list $[y_1 \dots y_n]$.
In other words, in this example $P$ is simply a \texttt{map} transformation:
\begin{equation}
[y_1 \dots y_n] = \mathbf{map}(\lambda~ x. f(x), [x_1 \dots x_n])
\label{eq:map}
\end{equation}

Noting that useful analytics tasks can be obtained by composing common functional transformations, for completeness we now show the framework in action on two other such transformations, namely \texttt{filter} and \texttt{reduce}.


\subsection{Filter}

A \texttt{filter} transformation is of the form:
\begin{equation}
[y_1 \dots y_m] = \mathbf{filter}(\lambda~ x. p(x), [x_1 \dots x_n])
\label{eq:filter}
\end{equation} 
where $p()$ is a predicate and $m\leq n$.
The following \texttt{T-filter} template captures the provenance structure:
\begin{quote}
\texttt{T-filter}: \\
\verb|entity(p, [prov:type = `prov:plan'])| \\
\verb|entity(:x), entity(:y)| \\
\verb|activity(:p), wasAssociatedWith(:p,_,p)| \\
\verb|used(:p,:x), wasGeneratedBy(:y,:p)| \\
\verb|wasDerivedFrom(:y,:x)| \\
\end{quote}
For  invocation (\ref{eq:filter}), a sidecar process $\mathtt{filter}^B$ must be defined to generate the following binding:
\begin{align*}
 B = \{ \langle & \texttt{:p} \leftarrow \texttt{gen}(\texttt{p},i), \texttt{:x} \leftarrow x_i, \texttt{:y} \leftarrow y_i  \rangle \\
                      & | i:1 \dots n \wedge p(x_i) = \texttt{True}\}
 \end{align*}
For this invocation, assuming for instance that $p$ evaluates to \texttt{True} only on $x_1, x_3$, $\mathit{apply}(\texttt{B-filter},\texttt{T-filter})$ results in the following PROV document:
\begin{quote}
\texttt{P-filter}: \\
\verb|entity(p, [prov:type = `prov:plan'])| \\
\verb|entity(x_1), entity(y_1)| \\
\verb|activity(p_1), wasAssociatedWith(p_1,_,p)| \\
\verb|used(p_1,x_1), wasGeneratedBy(y_1,p_1)| \\
\verb|wasDerivedFrom(y_1,x_1)| \\
\verb|entity(x_3), entity(y_3)| \\
\verb|activity(p_3), wasAssociatedWith(p_3,_,p)| \\
\verb|used(p_3,x_3), wasGeneratedBy(y_3,p_3)| \\
\verb|wasDerivedFrom(y_3,x_3)| \\
\end{quote}

 
 \subsection{Reduce}
The \texttt{reduce} operator maps a list of elements to a different type of new data structure:
\begin{equation}
y = \mathbf{reduce}(\lambda~ a,b. f(a,b), [x_1 \dots x_n])
\label{eq:reduce}
\end{equation} 
where $f(a,b)$ combines an accumulator $a$ with each new list input element $b$, producing a new value for $a$.
%
The corresponding template  is very similar to \texttt{T-filter} and \texttt{T1} above:
\begin{quote}
\texttt{T-reduce}: \\
\verb|entity(f, [prov:type = `prov:plan'])| \\
\verb|entity(:x), entity(:y)| \\
\verb|activity(:f), wasAssociatedWith(:f,_,f)| \\
\verb|used(:f,:x), wasGeneratedBy(:y,:f)| \\
\verb|wasDerivedFrom(:y,:x)|
\end{quote}
However, for \texttt{reduce} we need to express the fact that the single output $y$ is generated at the end of each invocation of $f(x_i)$, and is derived from each $x_i$.
For this, we set the assignment to \texttt{:y} to be the same in each of the bindings:
\begin{align*}
 B = \{ \langle & \texttt{:f} \leftarrow \texttt{gen}(\texttt{f},i), \texttt{:x} \leftarrow x_i, \texttt{:y} \leftarrow y  \rangle \\
                      & | i:1 \dots n\}
 \end{align*}
$\mathit{apply}(B,\texttt{T-reduce})$ produces the PROV document:
\begin{quote}
\texttt{P-reduce}: \\
\verb|entity(f, [prov:type = `prov:plan'])| \\
\verb|entity(x_1) ... entity(x_n)| \\
\verb|entity(y)| \\
\verb|activity(f_i), wasAssociatedWith(f_i,_,f)| \\
\verb|used(f_i,x_i), wasGeneratedBy(y,f_i)| \\
\verb|wasDerivedFrom(y,x_i)|
\end{quote}

Lastly, as an example of multi-input process, consider the case of a \texttt{join} operator, which is available on frameworks such as Spark to combine two inputs according to the usual join semantics:
$z = \mathbf{join}(x,y)$.

A straightforward template, following the pattern shown above, reflects the dependence of each element of $y$ from corresponding input pair $\langle x_i, y_i \rangle$, and is omitted for brevity.
 
\subsection{Sidecar process specification and execution}

So far we have introduced a framework that identifies the elements required to generate provenance from an arbitrary black box process $P$, and we have shown that it applies in particular to a common case of functional transformation.

To summarise, given $P$ with a known signature but otherwise unknown semantics, one should define: (i) a template $T$ and (ii) a binding generator process $P^B$.
The latter is a \textit{sidecar} process, in the sense that it must be executed for every invocation of the main process $P$.
%

The specifications of the template and sidecar process reflect the available knowledge of the input/output dependencies through $P$, even when $P$ itself maybe a black box.
If nothing at all is known about these dependencies, then clearly no provenance will be available and the framework will not help much.
On the other hand, the examples above show that, when $P$ has a well-defined semantics, it may be possible to capture enough of it in the template/sidecar pair, so that meaningful provenance can be generated. 

\note{as seen in the examples, the sidecar process may end up encoding some elements of the process logic itself!}

Either way, we believe that this simple framework can be useful to separate out the specification of a possibly complex function, for example, an entire analytics pipeline, from the specification, hopefully simpler, of the input/output dependencies through the function.
%
For instance, details of $P$ such as the lambda functions and predicates in our examples may not be available. 
In those cases, a template specification that is less precise than the ones given above, but possibly still useful, is one that only asserts derivation relationships, such as this simplified version of \texttt{T1}:
%
\begin{quote}
\verb|entity(:x), entity(:y), wasDerivedFrom(:y,:x)|
\end{quote}

A further concern is the implementation, deployment, and invocation of sidecar processes.
While we envision an abstract API to let $P^B$ access the current inputs and outputs of $P$ as required (see \ref{eq:B}), i.e., \texttt{getInputs()}, \texttt{getOutputs()}, 
the implementation of such API and the execution of $P^B$ must be supported by, and usually require ad hoc instrumentation of, $P$'s runtime environment. 
The problem of implementing provenance capture sub-systems has been broadly investigated in the past, for instance: 
(\cite{FRESCO}, \cite{noWorkflow}, \cite{Rlib}, \cite{Taverna provenance} \cite{VisTrails}, \cite{eSc provenance}, \cite{Kepler provenance}, \cite{SPADE}), and more recently for Spark \cite{Titian}.
The issue of adapting some of these systems to support sidecar processes is beyond the scope of this short paper.

\section{Application to stream data processing}  \label{sec:streaming}

So far we have  assumed that $P$ is a batch process.
However, the framework is perhaps most useful when $P$'s inputs and outputs are data streams, as then it can be used to generate a continuous stream of provenance assertions, to parallel the data stream produced by $P$.
%
To make a concrete case, we use the Spark Streaming model as reference \cite{spark-streaming-oreillybook}. 
Spark Streaming uses a \textit{micro-batch} architecture, whereby a streaming computation is achieved through a sequence of batch computations, each operating on a fragment of the stream defined by a configurable \textit{batch interval}. 
Each micro-batch consists of a finite sequence of data structures (RDDs, Spark's main data abstraction), known as a DStream.

For our purposes, we may view a DStream within each batch as a list of values, like those we have used in our batch examples. 
Thus, just as it allows Spark Streaming to reuse most of its RDD transformation and action operators, micro-batching also makes our framework reusable for generating provenance over streams.
Spark also supports \textit{windows} as application-friendly abstractions over streams. A window is simply a sequence of contiguous micro-batches, and a breakdown of the stream into windows is specified in the usual way, namely by a combination of window size (the number of micro-batches) and sliding duration.

In this setting, the template $T$ is defined so that it applies to each window, and the sidecar process executes once for each window, producing a \textit{stream of bindings} $B_1, B_2, \dots$.
Correspondingly, this triggers a sequence of calls: $\apply(B_1,T), \apply(B_2,T), \dots$, resulting in a stream of provenance statements alongside the corresponding data input and output streams.

Note that each of the batch process examples above can be mapped to a streaming setting without changes, as long as the scope of $P$ is restricted to a single window. 
In Spark, for instance, by default the scope of a \texttt{reduce} operation performed on a stream is the list of items within the window. 
This ensures that the \texttt{T-reduce} template in the previous Section, along with the corresponding sidecar process, produce the correct PROV documents, i.e., each new output $y$ produced upon completion of a window only depends on the inputs $x_1 \dots x_n$ within that window.
%
This limitation, however, can be removed by making Spark streaming computations stateful, that is, by having each micro-batch processing depend on a state that is defined but the previous batches.

We address the case of stateful transformations (either batch or streaming) next.


\section{Extension to stateful processes}  \label{sec:stateful}

All the examples given so far are of \textit{stateless} transformations, where the bindings generated by the sidecar processes only involve inputs and output values.
In contrast, \textit{stateful} transformations depend on some internal state, which may change as part of one invocation.
%
A simple example is a process $P$ that keeps a daily average of values for a stock, and generates an alert every time the current value is far from the current average.
As $P$ keeps an internal state that determines the values in the output stream, the provenance of each output is derived from the current state, as well from a specific input.
Furthermore, the state itself is updated at the end of each day, using the daily time series of stock values as well as the old state.

We are going to model this scenario by recording the provenance of the state itself, i.e., by considering the state as a PROV Entity and including it in the framework.

To illustrate the idea, let us assume that during a-one day window, $n$ stock values $[x_1 \dots x_n]$ are processed, and $m$ alerts $[y_1 \dots y_m]$ are generated.
We represent the sequence of states, that is, the daily averages, using a progression of Entity IDs: \texttt{s\_1}, \texttt{s\_2}, ... and the corresponding state-changing activities using 
\texttt{avg\_1}, \texttt{avg\_2}, ..., which are all associated with the same \texttt{avg} plan.
At the end of each day, a new PROV document would be generated. 
Thus, for instance at the end of day 10 the following document would be produced:
\begin{quote}
\texttt{P-state}:\\
\verb|# each output is derived from state s_10| \\
\verb|entity(y_1) ... entity(y_m)| \\
\verb|wasDerivedFrom(y_1,s_10) ... |\\
\verb|wasDerivedFrom(y_m,s_10) | \\
\verb|# new Entity is introduced for the new state| \\
\verb|entity(s_11)| \\
\verb|# the new state depends on all inputs |\\
\verb|# as well as the old state:| \\
\verb|entity(avg,_,[prov:type='prov:plan'])| \\
\verb|activity(avg_11)| \\
\verb|wasAssociatedWith(avg11,_,avg)|  \\ 
\verb|used(avg11,x_1) ... used(avg11,x_n)| \\
\verb|used(avg11,s_10)| \\
\verb|wasGeneratedBy(s_11,avg_10)| \\
\verb|wasDerivedFrom(s_11,x_1) ... | \\
\verb|wasDerivedFrom(s_11,x_n) ... | \\
\verb|wasDerivedFrom(s_11,s_10) |
\end{quote}
Here is a possible template from which such a document can be generated:
\begin{quote}
\texttt{P-state}:\\
\verb|entity(:y)| \\
\verb|wasDerivedFrom(:y,:s_old)|\\
%\verb|wasDerivedFrom(:y,:x)|\\
\verb|entity(avg,_,[prov:type='prov:plan'])| \\
\verb|activity(:avg_act)| \\
\verb|wasAssociatedWith(:avg_act,_,avg)|  \\ 
\verb|used(:avg_act,:x)| \\
\verb|used(:avg_act,:s_old)| \\
\verb|entity(:s_new)| \\
\verb|wasGeneratedBy(:s_new,:avg_act)| \\
\verb|wasDerivedFrom(:s_new,:x)| \\
\verb|wasDerivedFrom(:s_new,:s_old) |
\end{quote}

The corresponding binding $B$, generated by the sidecar process $P^B$ (recall that $B$ applies to each window) is now more involved than in past examples, as it needs to generate assignments for the \texttt{:x} and \texttt{:y} variables, as well as for the old and new states, \texttt{:s\_old} and \texttt{:s\_new}.
Furthermore, the process must also generate a new Entity ID for the state Entity each time it is invoked, to keep track of state changes, i.e. \texttt{s\_1}, \texttt{s\_2}, ...
For this, we assume it can persistently store a counter, which can be recalled (\texttt{getStateEntityID()}) as well as incremented \& returned (\texttt{incStateEntityID()}).

With these observations, we represent a binding as the union of three sets of assignments, each concerned with $y$, $x$, and the state, respectively:
\begin{align*}
B = & \{ \langle  \texttt{:y} \leftarrow y_i | i: 1 \dots m \rangle \} ~\cup \\
		 &  \{ \langle \texttt{:x} \leftarrow x_j,  \texttt{:avg\_act} \leftarrow \texttt{gen}(\texttt{avg},j) \rangle |~ j:1 \dots n  \} ~\cup \\
		 &\{ \langle \texttt{:s\_old} \leftarrow \texttt{getStateEntityId()}, \\
		 &  \texttt{:s\_new} \leftarrow \texttt{incStateEntityId()}  \rangle \}
 \end{align*}




\section{Extensions, implementation, and validation}

\note{the last example shows a limitation of the simple framework. hard to separate assignments that pertain to different sets of entities and relationships}

\note{1 binding / 1 template is limiting. generalise to multiple $\langle B, T \rangle$ pairs. So there are multiple templates, and the sidecar process generates one one binding per template (not all templates may have a binding, in which case they are \textit{inactive}.  Illustrate this to produce a better version of the last example. One binding may deal with state change, another just with the I/O.}.

\verb|# each y_i is also derived from some single input| \\
\verb|wasDerivedFrom(y_1,x_...) ... |\\
\verb|wasDerivedFrom(y_m,x_...)| 


\section{Appendix Title}

This is the text of the appendix, if you need one.

\acks

Acknowledgments, if needed.

% The 'abbrvnat' bibliography style is recommended.

\bibliographystyle{abbrvnat}

% The bibliography should be embedded for final submission.
%\bibliography{TBD}

%\begin{thebibliography}{}
%\softraggedright
%
%\bibitem[Smith et~al.(2009)Smith, Jones]{smith02}
%P. Q. Smith, and X. Y. Jones. ...reference text...
%
%\end{thebibliography}



\end{document}