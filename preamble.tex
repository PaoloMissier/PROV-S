\usepackage{amsmath}

\newcommand{\cL}{{\cal L}}

%\usepackage{biblatex}
%\usepackage[style=ieee]{biblatex}
%\AtEveryBibitem{%
%  \clearfield{issn} % Remove issn
%  \clearfield{doi} % Remove doi
%
%  \ifentrytype{online}{}{% Remove url except for @online
%    \clearfield{url}
%  }
%}

%\usepackage{pifont}
%\usepackage[T1]{fontenc} % this is just to get the natural encoding for the < > characters in text

\usepackage{color,soul}
\definecolor{highlight}{rgb}{1,1,0.6}
\definecolor{link}{rgb}{0.5,0.0,0.0}
\definecolor{cite}{rgb}{0.0,0.0,0.6}
\definecolor{url} {rgb}{0.3,0.0,0.3}
%\definecolor{grey}{rgb}{0.3,0.3,0.3}

\usepackage{algorithm}
\usepackage{algpseudocode}

\usepackage[hidelinks]{hyperref}
\hypersetup{
    colorlinks,
    linkcolor={cite},
    citecolor={cite},
    urlcolor ={cite}
}
\usepackage{enumitem}
\usepackage[british]{datetime2}

\usepackage{tabularx}
%\usepackage{array}
\usepackage{multirow}
\usepackage{booktabs}
%\newcolumntype{L}{>{\raggedright\arraybackslash}p{3cm}}
%\usepackage{dcolumn}
%\usepackage{multicol}

% % annotations environments % % 
\newcommand{\note}[1]{\textit{\textcolor{red}{\{#1\}}}}
\sethlcolor{highlight}


\def\ojoin{\setbox0=\hbox{$\bowtie$}%
  \rule[-.02ex]{.25em}{.4pt}\llap{\rule[\ht0]{.25em}{.4pt}}}
\def\leftouterjoin{\mathbin{\ojoin\mkern-5.8mu\bowtie}}
\def\rightouterjoin{\mathbin{\bowtie\mkern-5.8mu\ojoin}}
\def\fullouterjoin{\mathbin{\ojoin\mkern-5.8mu\bowtie\mkern-5.8mu\ojoin}}


%% PM Define authornote command for comments
\newcommand{\authornote}[1] {
    \begin{center}
        \framebox{
            {\begin{minipage}[t]{0.9\linewidth}
                \raggedright  \textbf{[PM]}~ \scriptsize #1 \normalsize
            \end{minipage}}
    }
    \end{center}
}

\usepackage{amssymb}
\let\oldemptyset\emptyset
\let\emptyset\varnothing

\usepackage{bm}
\usepackage{mathtools}
\def\filtcap{\mathrel{%
    \mathchoice{\FILTCAP}{\FILTCAP}{\scriptsize\FILTCAP}{\tiny\FILTCAP}%
}}
\def\FILTCAP{{%
    \setbox0\hbox{$\cap$}%
    \rlap{\hbox to \wd0{\hss$\circ$\hss}}\box0
}}

%\usepackage{datetime}
%\usepackage[mark]{gitinfo2}

\newcommand{\prov}{\mathit{prov}}
